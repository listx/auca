\documentclass[letterpaper,twoside,12pt]{article}
\usepackage[vmargin=1in,hmargin=1in]{geometry}
\usepackage[titles]{tocloft}

% make XeTeX work properly with the \- (discretionary intraword hyphen),
% \textsuperscript, and \textsubscript commands (also loads the 'xunicode'
% package and other packages to make XeTeX work better; it also loads the
% 'fontspec' package)
\usepackage{xltxtra}

% import predefined colors
\usepackage[x11names]{xcolor}

% shortcut command for footnotes
\newcommand*\fn[1]{\footnote{#1}}

\usepackage{listings}
\lstloadlanguages{Haskell}
\lstnewenvironment{code}
	{\lstset{tabsize=4}%
		\csname lst@SetFirstLabel\endcsname}
		{\csname lst@SaveFirstLabel\endcsname}
	\lstset{
		basicstyle=\small\ttfamily,
		frame=l,
		framerule=3pt,
		framexleftmargin=0.5em,
		framexrightmargin=0.5em,
		rulecolor=\color{Ivory3},
		backgroundcolor=\color{white},
		showstringspaces=false,
		flexiblecolumns=false,
    }

\renewcommand*\oldstylenums{\addfontfeature{Numbers={OldStyle}}}
\newcommand*\ct[1]{\texttt{\color{DeepPink3}#1}}
\setromanfont[Mapping=tex-text,Ligatures=Common]{Charis SIL}
\setsansfont[Mapping=tex-text,Ligatures=Common,Scale=0.8]{DejaVu Sans Condensed}
\setmonofont[Scale=0.8]{DejaVu Sans Mono}

% Make all monospaced text bolded.
\let\temptt\texttt
\renewcommand\texttt[1]{\textbf{\temptt{#1}}}

% intra-document hyperlinks
\usepackage[colorlinks,linkcolor=blue]{hyperref}

\begin{document}
\renewcommand{\thefootnote}{\fnsymbol{footnote}}
\title{\Huge Auca Source Manual}
\author{Linus Arver\footnotemark[1]\ \footnotemark[2]
\\
\texttt{GIT-COMMIT-DESC}\footnotemark[4]
}
\date{\texttt{GIT-COMMIT-DATE}}
\footnotetext[1]{Email: \texttt{X@Y.Z}, where Z is \texttt{edu}, Y is \texttt{ucla}, and X is \texttt{linus}.}
\footnotetext[2]{Website: \texttt{\url{http://listx.github.io}}.}
\footnotetext[4]{This document is generated from the sources from the latest commit. The full hash of this commit is \href{http://www.github.com/listx/auca/commit/GIT-COMMIT-HASH}{\texttt{GIT-COMMIT-HASH}}.}
\maketitle{}
\renewcommand{\thefootnote}{\arabic{footnote}}

\tableofcontents
\section{Introduction}

\ct{auca} is a program that automatically executes an arbitrary command based on the modification of a file or set of files.

\subsection{How to Read This Manual}

The general format is to show the raw source code first, followed by commentary on what the just-shown block of code does.
The idea is to try to read the source code first, and then have it explained in detail later.
Whenever the commentary says ``this block of code'' or ``here'', it is referring to the block of code directly above it.

\documentclass[letterpaper,twoside,12pt]{article}
\usepackage[vmargin=1in,hmargin=1in]{geometry}
\usepackage[titles]{tocloft}

% make XeTeX work properly with the \- (discretionary intraword hyphen),
% \textsuperscript, and \textsubscript commands (also loads the 'xunicode'
% package and other packages to make XeTeX work better; it also loads the
% 'fontspec' package)
\usepackage{xltxtra}

% import predefined colors
\usepackage[x11names]{xcolor}

% shortcut command for footnotes
\newcommand*\fn[1]{\footnote{#1}}

\usepackage{listings}
\lstloadlanguages{Haskell}
\lstnewenvironment{code}
	{\lstset{tabsize=4}%
		\csname lst@SetFirstLabel\endcsname}
		{\csname lst@SaveFirstLabel\endcsname}
	\lstset{
		basicstyle=\small\ttfamily,
		frame=lines,
		framexleftmargin=0.5em,
		framexrightmargin=0.5em,
		backgroundcolor=\color{LemonChiffon1},
		showstringspaces=false,
		flexiblecolumns=false,
    }

\renewcommand*\oldstylenums{\addfontfeature{Numbers={OldStyle}}}
\newcommand*\ct[1]{\texttt{\color{DeepPink3}#1}}
\setromanfont[Mapping=tex-text,Ligatures=Common]{Charis SIL}
\setsansfont[Mapping=tex-text,Ligatures=Common,Scale=0.8]{DejaVu Sans Condensed}
\setmonofont[Scale=0.8]{DejaVu Sans Mono}

% intra-document hyperlinks
\usepackage[colorlinks,linkcolor=blue]{hyperref}

\begin{document}
\title{The \texttt{auca} Source Manual}
\author{Linus Arver\\
\\
ver. \href{http://www.github.com/listx/auca/commit/GIT-COMMIT-HASH}{\texttt{GIT-COMMIT-DESC}}
}
\date{\texttt{GIT-COMMIT-DATE}}
\maketitle
\tableofcontents
\section{Introduction}

\ct{auca} is a program that automatically executes an arbitrary command based on the modification of a file or set of files.

\subsection{How to Read This Manual}

The general format is to show the raw source code first, followed by commentary on what the just-shown block of code does.
The idea is to try to read the source code first, and then have it explained in detail later.
Whenever the commentary says ``this block of code'' or ``here'', it is referring to the block of code directly above it.

\documentclass[letterpaper,twoside,12pt]{article}
\usepackage[vmargin=1in,hmargin=1in]{geometry}
\usepackage[titles]{tocloft}

% make XeTeX work properly with the \- (discretionary intraword hyphen),
% \textsuperscript, and \textsubscript commands (also loads the 'xunicode'
% package and other packages to make XeTeX work better; it also loads the
% 'fontspec' package)
\usepackage{xltxtra}

% import predefined colors
\usepackage[x11names]{xcolor}

% shortcut command for footnotes
\newcommand*\fn[1]{\footnote{#1}}

\usepackage{listings}
\lstloadlanguages{Haskell}
\lstnewenvironment{code}
	{\lstset{tabsize=4}%
		\csname lst@SetFirstLabel\endcsname}
		{\csname lst@SaveFirstLabel\endcsname}
	\lstset{
		basicstyle=\small\ttfamily,
		frame=lines,
		framexleftmargin=0.5em,
		framexrightmargin=0.5em,
		backgroundcolor=\color{LemonChiffon1},
		showstringspaces=false,
		flexiblecolumns=false,
    }

\renewcommand*\oldstylenums{\addfontfeature{Numbers={OldStyle}}}
\newcommand*\ct[1]{\texttt{\color{DeepPink3}#1}}
\setromanfont[Mapping=tex-text,Ligatures=Common]{Charis SIL}
\setsansfont[Mapping=tex-text,Ligatures=Common,Scale=0.8]{DejaVu Sans Condensed}
\setmonofont[Scale=0.8]{DejaVu Sans Mono}

% intra-document hyperlinks
\usepackage[colorlinks,linkcolor=blue]{hyperref}

\begin{document}
\title{The \texttt{auca} Source Manual}
\author{Linus Arver\\
\\
ver. \href{http://www.github.com/listx/auca/commit/GIT-COMMIT-HASH}{\texttt{GIT-COMMIT-DESC}}
}
\date{\texttt{GIT-COMMIT-DATE}}
\maketitle
\tableofcontents
\section{Introduction}

\ct{auca} is a program that automatically executes an arbitrary command based on the modification of a file or set of files.

\subsection{How to Read This Manual}

The general format is to show the raw source code first, followed by commentary on what the just-shown block of code does.
The idea is to try to read the source code first, and then have it explained in detail later.
Whenever the commentary says ``this block of code'' or ``here'', it is referring to the block of code directly above it.

\documentclass[letterpaper,twoside,12pt]{article}
\usepackage[vmargin=1in,hmargin=1in]{geometry}
\usepackage[titles]{tocloft}

% make XeTeX work properly with the \- (discretionary intraword hyphen),
% \textsuperscript, and \textsubscript commands (also loads the 'xunicode'
% package and other packages to make XeTeX work better; it also loads the
% 'fontspec' package)
\usepackage{xltxtra}

% import predefined colors
\usepackage[x11names]{xcolor}

% shortcut command for footnotes
\newcommand*\fn[1]{\footnote{#1}}

\usepackage{listings}
\lstloadlanguages{Haskell}
\lstnewenvironment{code}
	{\lstset{tabsize=4}%
		\csname lst@SetFirstLabel\endcsname}
		{\csname lst@SaveFirstLabel\endcsname}
	\lstset{
		basicstyle=\small\ttfamily,
		frame=lines,
		framexleftmargin=0.5em,
		framexrightmargin=0.5em,
		backgroundcolor=\color{LemonChiffon1},
		showstringspaces=false,
		flexiblecolumns=false,
    }

\renewcommand*\oldstylenums{\addfontfeature{Numbers={OldStyle}}}
\newcommand*\ct[1]{\texttt{\color{DeepPink3}#1}}
\setromanfont[Mapping=tex-text,Ligatures=Common]{Charis SIL}
\setsansfont[Mapping=tex-text,Ligatures=Common,Scale=0.8]{DejaVu Sans Condensed}
\setmonofont[Scale=0.8]{DejaVu Sans Mono}

% intra-document hyperlinks
\usepackage[colorlinks,linkcolor=blue]{hyperref}

\begin{document}
\title{The \texttt{auca} Source Manual}
\author{Linus Arver\\
\\
ver. \href{http://www.github.com/listx/auca/commit/GIT-COMMIT-HASH}{\texttt{GIT-COMMIT-DESC}}
}
\date{\texttt{GIT-COMMIT-DATE}}
\maketitle
\tableofcontents
\input{intro.tex}
\input{../src/auca.lhs}
\input{../src/AUCA/Option.lhs}
\input{../src/AUCA/Core.lhs}
\input{../src/AUCA/Util.lhs}
\input{../src/AUCA/Meta.lhs}
\end{document}

\input{../src/AUCA/Option.lhs}
\input{../src/AUCA/Core.lhs}
\input{../src/AUCA/Util.lhs}
\input{../src/AUCA/Meta.lhs}
\end{document}

\input{../src/AUCA/Option.lhs}
\input{../src/AUCA/Core.lhs}
\input{../src/AUCA/Util.lhs}
\input{../src/AUCA/Meta.lhs}
\end{document}

\input{../src/AUCA/Option.lhs}
\input{../src/AUCA/Core.lhs}
\input{../src/AUCA/Util.lhs}
\input{../src/AUCA/Meta.lhs}
\end{document}
